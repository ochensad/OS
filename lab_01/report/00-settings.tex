\usepackage{cmap} % Улучшенный поиск русских слов в полученном pdf-файле
\usepackage[T2A]{fontenc} % Поддержка русских букв
\usepackage[utf8]{inputenc} % Кодировка utf8
\usepackage[english,russian]{babel} % Языки: русский, английский
%\usepackage{pscyr} % Нормальные шрифты
\usepackage{enumitem} % Настройка оформления списков

\usepackage[14pt]{extsizes} % Задание 14-размера шрифта

\usepackage{caption} % Подпись картинок и таблиц
\captionsetup{labelsep=endash} % Разделитель между номером и текстом краткое тире и пробел
\captionsetup[figure]{name={Рисунок}} % Изменяет имя для всех фигур на "Рисунок"

\usepackage{amsmath} % Что-то связанное с математикой

\usepackage[left=3cm,right=1.5cm,top=2cm,bottom=2cm]{geometry} % Задание геометрии листа

\usepackage{titlesec} % Оформление заголовков
\titleformat{\section}
	{\normalsize\bfseries}
	{\thesection}
	{1em}{}
\titlespacing*{\chapter}{0pt}{-30pt}{8pt}
\titlespacing*{\section}{\parindent}{*4}{*4}
\titlespacing*{\subsection}{\parindent}{*4}{*4}

\usepackage{titlesec}
\titleformat{\chapter}{\LARGE\bfseries}{\thechapter}{20pt}{\LARGE\bfseries}
\titleformat{\section}{\Large\bfseries}{\thesection}{20pt}{\Large\bfseries}

\usepackage{setspace}
\onehalfspacing % Полуторный интервал

\frenchspacing
\usepackage{indentfirst} % Красная строка

\usepackage{listings} % Оформление листингов
\usepackage{xcolor} % Добавление цветов 

\lstdefinestyle{asm}{ % Опеределение стиля 
	language={[x86masm]Assembler},
	backgroundcolor=\color{white},
	basicstyle=\footnotesize\ttfamily,
	keywordstyle=\color{blue},
	stringstyle=\color{red},
	commentstyle=\color{gray},
	numbers=left,
	numberstyle=\tiny,
	stepnumber=1,
	numbersep=5pt,
	frame=single,
	tabsize=4,
	captionpos=b,
	breaklines=true,
	breakatwhitespace=true
}

\usepackage{pgfplots} % Построение графиков
\usetikzlibrary{datavisualization}
\usetikzlibrary{datavisualization.formats.functions}

\usepackage{graphicx} % Вставка рисунков

\newcommand{\img}[3] {
	\begin{figure}[h!]
		\center{\includegraphics[height=#1]{img/#2}}
		\caption{#3}
		\label{img:#2}
	\end{figure}
}

%\newcommand{\boximg}[3] {
%	\begin{figure}[h]
%		\center{\fbox{\includegraphics[height=#1]{inc/img/#2}}}
%		\caption{#3}
%		\label{img:#2}
%	\end{figure}
%}

\usepackage[justification=centering]{caption} % Настройка подписей float объектов

\usepackage[unicode,pdftex]{hyperref} % Ссылки в pdf
\hypersetup{hidelinks}

\usepackage{csvsimple}



%\newcommand{\code}[1]{\texttt{#1}}
